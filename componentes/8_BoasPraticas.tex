\newpage
\section{Boas Práticas}

Adotar boas práticas no uso do Git é essencial para garantir a organização, a rastreabilidade e a eficiência dos projetos colaborativos. Nesta seção, abordamos recomendações para escrever mensagens de commit claras, organizar repositórios, evitar erros comuns e implementar metodologias de desenvolvimento como o GitFlow.

\subsection{Escrevendo Bons Commit Messages}

Mensagens de commit bem escritas facilitam o entendimento do histórico do projeto e ajudam toda a equipe a acompanhar as alterações. Recomendações:
\begin{itemize}
    \item Seja objetivo e claro: descreva o que foi alterado e por quê.
    \item Evite mensagens genéricas como \texttt{"Update"} ou \texttt{"Correções"}.
    \item Se necessário, adicione uma descrição mais detalhada após a primeira linha.
\end{itemize}

\subsection{Organizando Repositórios}

Um repositório bem organizado facilita a navegação e o entendimento do projeto. Dicas:
\begin{itemize}
    \item Mantenha uma estrutura de pastas lógica (ex: \texttt{src/}, \texttt{docs/}, \texttt{data/}).
    \item Utilize arquivos \texttt{README.md} para documentar o objetivo do projeto e instruções de uso.
    \item Adote o \texttt{.gitignore} para evitar versionar arquivos desnecessários ou temporários.
    \item Padronize nomes de arquivos e pastas.
\end{itemize}

\subsection{Erros Comuns a Evitar}

Alguns erros podem comprometer a organização e a colaboração no projeto:
\begin{itemize}
    \item Versionar arquivos grandes ou gerados automaticamente (ex: \texttt{.mat}, \texttt{.exe}, \texttt{.log}).
    \item Realizar commits diretamente na branch principal sem revisão.
    \item Não atualizar o repositório local antes de iniciar novas tarefas.
    \item Mensagens de commit vagas ou sem contexto.
    \item Não resolver conflitos de merge corretamente.
\end{itemize}

% \subsection{Assinatura de Commits (Opcional)}

% Assinar commits com uma chave GPG garante autenticidade e segurança, especialmente em projetos públicos ou críticos. Para configurar:
% \begin{lstlisting}[style=shellstyle]
% git config --global user.signingkey <ID-da-chave>
% git commit -S -m "Mensagem assinada"
% \end{lstlisting}
% Mais detalhes podem ser encontrados na documentação oficial do Git: \url{https://git-scm.com/book/en/v2/Git-Tools-Signing-Your-Work}

\subsection{GitFlow - Metodologia de Desenvolvimento Colaborativo}

O \textbf{GitFlow} é uma metodologia que define um fluxo de trabalho estruturado para equipes que usam Git. Ele organiza o desenvolvimento em diferentes tipos de branches, cada um com uma finalidade específica:
\begin{itemize}
    \item \texttt{main} (ou \texttt{master}): branch principal, sempre estável.
    \item \texttt{develop}: branch de desenvolvimento, onde novas funcionalidades são integradas antes de serem lançadas.
    \item \texttt{feature/*}: branches para desenvolvimento de novas funcionalidades.
    \item \texttt{release/*}: branches para preparação de novas versões.
    \item \texttt{hotfix/*}: branches para correção rápida de bugs em produção.
\end{itemize}

O GitFlow facilita o gerenciamento de múltiplas tarefas simultâneas, garante que o código principal permaneça estável e organiza o processo de lançamento de versões.

Para implementar o GitFlow, existem extensões e ferramentas que automatizam o processo. O documento original do GitFlow pode ser acessado em: \url{https://nvie.com/posts/a-successful-git-branching-model/}

Adotar o GitFlow em projetos colaborativos aumenta a produtividade, reduz conflitos e torna o ciclo de desenvolvimento mais previsível.

\begin{quote}
\textbf{Nota:}
A leitura do documento oficial do GitFlow é fortemente recomendada. Uma boa compreensão dessa metodologia colaborativa esclarece o funcionamento de um repositório Git bem estruturado e contribui para o sucesso de projetos versionados em equipe.
\end{quote}

