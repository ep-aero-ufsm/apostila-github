\newpage
\section{Colaboração e Trabalho em Equipe}

O Git é uma ferramenta poderosa para facilitar o trabalho colaborativo em projetos, permitindo que equipes trabalhem simultaneamente em diferentes partes de um projeto sem conflitos e com total rastreabilidade. Nesta seção, exploraremos como o Git pode ser utilizado para gerenciar contribuições de diferentes membros da equipe, revisar código e adotar boas práticas para garantir a eficiência e a organização do trabalho em grupo.

\subsection{Pull Requests}

O Pull Request (PR) é uma funcionalidade oferecida por plataformas como GitHub e GitLab que permite que um colaborador proponha alterações para um repositório. Ele é amplamente utilizado em projetos colaborativos para integrar contribuições de forma controlada e revisada.

O fluxo básico de um Pull Request é o seguinte:
\begin{enumerate}
    \item O colaborador cria uma nova branch para desenvolver uma funcionalidade ou corrigir um problema.
    \item Após realizar as alterações e fazer os commits, o colaborador envia a branch para o repositório remoto.
    \item No repositório remoto, o colaborador abre um Pull Request, descrevendo as alterações realizadas e o motivo delas.
    \item Outros membros da equipe revisam o Pull Request, sugerem melhorias e aprovam as alterações.
    \item Após a aprovação, as alterações são integradas à branch principal (\texttt{main} ou \texttt{master}) por meio de um merge.
\end{enumerate}

Os Pull Requests são uma excelente forma de garantir que todas as alterações sejam revisadas antes de serem integradas ao projeto, promovendo a qualidade do código e a colaboração entre os membros da equipe.

\subsection{Revisão de Código}

A revisão de código é uma etapa essencial no trabalho colaborativo, pois permite que os membros da equipe avaliem as contribuições uns dos outros, identifiquem erros, sugiram melhorias e garantam a consistência do projeto. Durante a revisão de um Pull Request, os revisores devem:
\begin{itemize}
    \item Verificar se o código segue os padrões e boas práticas estabelecidos pela equipe.
    \item Garantir que as alterações não introduzam erros ou quebras no projeto.
    \item Testar as novas funcionalidades ou correções, quando aplicável.
    \item Sugerir melhorias para tornar o código mais eficiente, legível ou organizado.
\end{itemize}

A revisão de código não deve ser vista como uma crítica pessoal, mas como uma oportunidade de aprendizado e melhoria contínua para toda a equipe. Ferramentas como comentários em Pull Requests no GitHub ou GitLab tornam esse processo mais eficiente e colaborativo.

\subsection{Boas Práticas em Projetos de Equipe}

Para garantir o sucesso de um projeto colaborativo, é importante adotar boas práticas no uso do Git. Algumas recomendações incluem:
\begin{itemize}
    \item \textbf{Criar branches para cada tarefa:} Cada funcionalidade ou correção deve ser desenvolvida em uma branch separada, com um nome descritivo, como \texttt{feature/simulacao} ou \texttt{bugfix/corrigir-grafico}.
    \item \textbf{Escrever mensagens de commit claras:} Cada commit deve ter uma mensagem que explique de forma objetiva o que foi alterado, como \texttt{"Adiciona função para calcular resistência do ar"}.
    \item \textbf{Sincronizar frequentemente:} Antes de iniciar uma nova tarefa, atualize sua branch local com as alterações mais recentes da branch principal para evitar conflitos.
    \item \textbf{Resolver conflitos de forma colaborativa:} Quando conflitos de versão ocorrerem, discuta com os membros da equipe para decidir a melhor forma de resolvê-los.
    \item \textbf{Revisar e testar antes de integrar:} Antes de fazer o merge de uma branch, certifique-se de que todas as alterações foram revisadas e testadas.
\end{itemize}

Essas práticas ajudam a manter o projeto organizado, reduzir erros e promover um ambiente de trabalho colaborativo e produtivo.

% \subsection{Exemplo Prático: Contribuindo para um Repositório Compartilhado}

% %% ber - NECESSITA SER TOTALMENTE REFEITO E PENSADO, isso é um exemplo do gpt

% Vamos considerar um exemplo prático de colaboração em um projeto de simulação de voo atmosférico. Suponha que você e sua equipe estão desenvolvendo um modelo no MATLAB/Simulink e precisam adicionar uma nova funcionalidade para calcular a resistência do ar.

% \subsubsection*{Passo 1: Criar uma branch para a tarefa}

% No terminal, crie uma nova branch para desenvolver a funcionalidade:
% \begin{lstlisting}[style=shellstyle]
% git checkout -b feature/resistencia-ar
% \end{lstlisting}

% \subsubsection*{Passo 2: Fazer as alterações e commits}

% Implemente a funcionalidade no código e registre as alterações com commits claros:
% \begin{lstlisting}[style=shellstyle]
% git add simulacao.m
% git commit -m "Adiciona função para calcular resistência do ar"
% \end{lstlisting}

% \subsubsection*{Passo 3: Enviar a branch para o repositório remoto}

% Envie a branch para o repositório remoto para compartilhar seu trabalho:
% \begin{lstlisting}[style=shellstyle]
% git push origin feature/resistencia-ar
% \end{lstlisting}

% \subsubsection*{Passo 4: Abrir um Pull Request}

% No GitHub ou GitLab, abra um Pull Request para a branch \texttt{feature/resistencia-ar}, descrevendo as alterações realizadas.

% \subsubsection*{Passo 5: Revisar e integrar as alterações}

% Os membros da equipe revisam o Pull Request, sugerem melhorias e, após a aprovação, fazem o merge da branch na branch principal:
% \begin{lstlisting}[style=shellstyle]
% git checkout main
% git merge feature/resistencia-ar
% \end{lstlisting}

% \subsubsection*{Passo 6: Atualizar o repositório local}

% Após o merge, atualize seu repositório local para refletir as alterações mais recentes:
% \begin{lstlisting}[style=shellstyle]
% git pull origin main
% \end{lstlisting}

% Com esse fluxo, a equipe pode colaborar de forma eficiente, garantindo que todas as contribuições sejam revisadas, testadas e integradas de maneira