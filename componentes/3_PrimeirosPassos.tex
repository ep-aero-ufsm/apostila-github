\newpage
\section{Primeiros Passos com Git}
\subsection{Criando seu Primeiro Repositório}

Um repositório é o local onde o Git armazena todo o histórico de alterações de um projeto. Ele pode ser \textbf{local} (no seu computador) ou \textbf{remoto} (em plataformas como GitHub ou GitLab). Criar seu primeiro repositório é o passo inicial para começar a versionar arquivos e acompanhar a evolução de seus projetos.

Para estudantes de Engenharia Aeroespacial, um exemplo típico seria um projeto de simulação de voo, com arquivos MATLAB/Simulink, planilhas de dados e relatórios técnicos. Ao criar um repositório, todas essas informações podem ser organizadas, versionadas e recuperadas facilmente.

\subsubsection*{Passo a Passo: Criando um Repositório Local}

1. \textbf{Escolha ou crie uma pasta para seu projeto}  
   Por exemplo, crie uma pasta chamada \texttt{simulacao\_orbital} no seu computador.

2. \textbf{Abra o terminal}  
   - No macOS ou Linux, use o \textit{Terminal}.  
   - No Windows, abra o \textbf{Git Bash}.  
   - Alternativamente, você pode usar o terminal integrado do \textbf{Visual Studio Code}.

3. \textbf{Navegue até a pasta do projeto}


\begin{lstlisting}[style=shellstyle]
cd /caminho/para/simulacao_orbital
\end{lstlisting}


\textit{Comentário: O comando \texttt{cd} significa "change directory" e serve para trocar o diretório de trabalho atual no terminal. Assim, você garante que os próximos comandos do Git serão aplicados à pasta correta.}

4. \textbf{Inicialize o repositório Git}  
   Esse comando cria um repositório Git local na pasta atual, gerando a pasta oculta \texttt{.git} que armazenará o histórico do projeto:
\begin{lstlisting}[style=shellstyle]
git init
\end{lstlisting}

5. \textbf{Verifique se o repositório foi criado}  
   Execute:
\begin{lstlisting}[style=shellstyle]
git status
\end{lstlisting}
O Git mostrará que você está em um repositório vazio e pronto para adicionar arquivos.


\subsection{Rastreando Arquivos}

Após criar seu repositório, o próximo passo é informar ao Git quais arquivos você deseja versionar. Esse processo é chamado de \textbf{staging}, ou preparação, e é realizado com o comando \texttt{git add}. Ele permite que você escolha exatamente quais alterações serão incluídas no próximo commit.

\subsubsection*{Adicionando arquivos ao repositório}

Para adicionar todos os arquivos da pasta do projeto, execute:

\begin{lstlisting}[style=shellstyle]
git add .
\end{lstlisting}

\noindent
O ponto (\texttt{.}) indica que todos os arquivos e subpastas do diretório atual serão rastreados.  

Se você quiser adicionar apenas arquivos específicos, use:

\begin{lstlisting}[style=shellstyle]
git add simulacao.m relatorio.pdf
\end{lstlisting}

\subsubsection*{Ignorando arquivos desnecessários}

Em projetos de engenharia aeroespacial, é comum gerar arquivos temporários ou grandes que não precisam ser versionados, como:  
- Logs de simulação (\texttt{*.log})  
- Arquivos de saída do Simulink (\texttt{*.slx~})  
- Dados intermediários ou temporários (\texttt{*.mat}, \texttt{*.tmp})

Para evitar que esses arquivos sejam rastreados, crie um arquivo chamado \texttt{.gitignore} na raiz do projeto e adicione as regras:

\begin{lstlisting}[style=shellstyle]
*.log
*.slx~
*.tmp
*.mat
\end{lstlisting}

Dessa forma, o Git ignorará automaticamente esses arquivos ao executar \texttt{git add}.

\subsection{Fazendo seu Primeiro Commit}

Depois de adicionar os arquivos desejados à área de staging, o próximo passo é registrar essas alterações no repositório através de um \textbf{commit}. Um commit funciona como um “ponto de restauração” no histórico do projeto, permitindo recuperar versões anteriores a qualquer momento.

\subsubsection*{Criando um commit}

Para fazer o primeiro commit, execute:

\begin{lstlisting}[style=shellstyle]
git commit -m "Primeiro commit: estrutura inicial do projeto"
\end{lstlisting}

\noindent
A opção \texttt{-m} permite adicionar uma mensagem descritiva para o commit. Mensagens claras ajudam você e sua equipe a entender rapidamente o que foi alterado em cada commit.

\subsubsection*{Verificando o commit}

Após o commit, você pode conferir o histórico do repositório:

\begin{lstlisting}[style=shellstyle]
git log
\end{lstlisting}

O Git exibirá informações como:
\begin{itemize}
    \item Identificador único do commit (hash)  
    \item Autor do commit  
    \item Data e hora do commit  
    \item Mensagem do commit
\end{itemize}