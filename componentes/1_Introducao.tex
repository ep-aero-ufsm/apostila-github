\section{Introdução Geral}

O desenvolvimento de projetos de engenharia, especialmente na área aeroespacial, envolve o trabalho com múltiplos arquivos, modelos, códigos e documentações que são constantemente atualizados e aprimorados. Nesse cenário, manter a organização, a rastreabilidade e a colaboração entre os membros de um time é um grande desafio. Para lidar com essas demandas, o uso de sistemas de \textbf{controle de versão} tornou-se indispensável, permitindo registrar alterações, restaurar versões anteriores, integrar contribuições de diferentes colaboradores e garantir a integridade do projeto.

O \textbf{Git} é atualmente a ferramenta de controle de versão mais utilizada no mundo acadêmico e industrial. Desenvolvido inicialmente por Linus Torvalds para gerenciar o código-fonte do sistema operacional Linux, o Git tornou-se um padrão na indústria de tecnologia devido à sua eficiência, flexibilidade e suporte à colaboração distribuída. Diferente de sistemas tradicionais de controle de versão, o Git permite que cada desenvolvedor possua uma cópia completa do histórico do projeto, o que aumenta a segurança, a autonomia e a velocidade no desenvolvimento.

Além disso, o Git é amplamente integrado a plataformas como \textit{GitHub}, \textit{GitLab} e \textit{Bitbucket}, que oferecem recursos adicionais para colaboração, revisão de código, gestão de tarefas e integração com ferramentas de automação. Essas plataformas são utilizadas não apenas por empresas de tecnologia, mas também por instituições acadêmicas e equipes de pesquisa, tornando o domínio do Git uma habilidade essencial para qualquer estudante ou profissional que deseja atuar em projetos modernos e colaborativos.

O objetivo central desta apostila é introduzir o Git como uma ferramenta prática e poderosa para organização, versionamento e colaboração no desenvolvimento de projetos. Ao longo do material, os leitores serão guiados por conceitos teóricos fundamentais e exemplos práticos, de forma que, ao final, estejam aptos a criar, gerenciar e compartilhar seus próprios repositórios, além de contribuir para projetos em equipe com segurança e eficiência.

\subsection{Objetivo da Apostila}

Esta apostila foi desenvolvida pela Escola Piloto de Engenharia Aeroespacial da Universidade Federal de Santa Maria com o objetivo de apresentar, de forma clara e progressiva, os fundamentos teóricos e práticos do \textbf{Git}. O material busca capacitar os leitores para:

\begin{itemize}
    \item Compreender os conceitos essenciais de controle de versão e a importância do Git no gerenciamento de projetos;
    \item Criar e manipular repositórios locais, registrando o histórico de alterações por meio de commits;
    \item Trabalhar com repositórios remotos, utilizando plataformas como GitHub e GitLab para compartilhar e integrar projetos;
    \item Aplicar técnicas de \textit{branching} e \textit{merging} para organizar fluxos de trabalho e gerenciar o desenvolvimento colaborativo;
    \item Adotar boas práticas para escrever mensagens de commit claras, organizar arquivos e estruturar projetos de forma eficiente;
    \item Desenvolver autonomia para lidar com conflitos de versão, entender fluxos de trabalho de equipes e colaborar de maneira produtiva.
\end{itemize}

Mais do que ensinar comandos e procedimentos, esta apostila busca desenvolver uma \textbf{mentalidade de versionamento}, na qual os estudantes aprendem a pensar na evolução de um projeto como um processo estruturado, seguro e colaborativo. Ao final, espera-se que os leitores tenham adquirido não apenas conhecimento técnico, mas também uma visão prática de como o Git se integra ao ciclo de desenvolvimento de sistemas e aplicações na engenharia aeroespacial.

\subsection{Público-Alvo}

Esta apostila foi elaborada especialmente para \textbf{estudantes dos primeiros semestres do curso de Engenharia Aeroespacial} da Universidade Federal de Santa Maria, mas também pode ser útil para alunos de outros cursos de engenharia, ciência da computação e áreas afins. O material parte do princípio de que o leitor está em fase inicial de contato com ferramentas de desenvolvimento e colaboração, não exigindo conhecimentos avançados de programação ou experiência prévia com controle de versão.

Para os estudantes de engenharia aeroespacial, o domínio do Git é particularmente relevante, pois a área envolve projetos multidisciplinares que integram diferentes componentes, como simulações, modelagem matemática, controle de sistemas, análise de dados e desenvolvimento de software embarcado. Projetos colaborativos, sejam eles acadêmicos, laboratoriais ou de pesquisa, tornam-se mais organizados, seguros e eficientes com o uso do Git, permitindo gerenciar versões, acompanhar o progresso e integrar contribuições de diferentes integrantes da equipe.

Além disso, aprender Git desde os primeiros semestres proporciona aos estudantes uma base sólida para enfrentar desafios futuros, seja no desenvolvimento de trabalhos acadêmicos, na execução de projetos de pesquisa, na participação em equipes de competição tecnológica ou na preparação para estágios e oportunidades no setor aeroespacial.

\subsection{Estrutura da Apostila}

A apostila foi organizada de forma progressiva, abordando desde os conceitos fundamentais até práticas mais avançadas de versionamento e colaboração. Sua estrutura é dividida em seções principais:

\begin{itemize}
    \item \textbf{Conceitos Básicos de Git:} Introdução ao controle de versão, instalação e configuração inicial.
    \item \textbf{Primeiros Passos com Git:} Criação de repositórios, rastreamento de arquivos, commits e fluxo de trabalho.
    \item \textbf{Trabalhando com Repositórios Remotos:} Integração com plataformas como GitHub, envio e atualização de projetos.
    \item \textbf{Colaboração e Trabalho em Equipe:} Revisão de código, pull requests e boas práticas de contribuição.
    \item \textbf{Branching e Merging:} Criação e gerenciamento de branches, resolução de conflitos e organização do desenvolvimento.
    \item \textbf{Boas Práticas:} Estratégias para mensagens de commit, organização de repositórios e metodologias como GitFlow.
    \item \textbf{Exercícios Práticos:} Atividades guiadas para consolidar os conhecimentos, incluindo um mini-projeto colaborativo.
    \item \textbf{Recursos e Referências:} Fontes de estudo e materiais de apoio para aprofundar o aprendizado.
\end{itemize}

Com essa abordagem, espera-se que o leitor desenvolva não apenas o domínio técnico sobre os comandos do Git, mas também a compreensão de como aplicar essas ferramentas para melhorar a organização, a produtividade e a qualidade de projetos individuais e colaborativos.