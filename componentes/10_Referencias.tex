\newpage
\section{Recursos e Referências}

Para aprofundar seus conhecimentos em Git e trabalho colaborativo, é importante consultar materiais de referência e recursos oficiais. Abaixo estão algumas recomendações úteis para estudo e consulta durante o desenvolvimento de projetos.

\subsection{Documentação Oficial do Git}

A documentação oficial do Git é completa e detalhada, abordando desde conceitos básicos até comandos avançados. Recomenda-se a leitura para esclarecer dúvidas e explorar funcionalidades adicionais.

\begin{itemize}
    \item Site oficial: \url{https://git-scm.com/doc}
    \item Livro gratuito: \url{https://git-scm.com/book/en/v2}
\end{itemize}

\subsection{GitHub Learning Lab}

O GitHub Learning Lab oferece cursos interativos gratuitos sobre Git, GitHub e colaboração em projetos. Os exercícios são práticos e guiados, ideais para iniciantes e para quem deseja aprimorar suas habilidades.

\begin{itemize}
    \item Acesse: \url{https://github.com/apps/github-learning-lab}
\end{itemize}

\subsection{Cheat Sheets Recomendados}

Cheat sheets são resumos práticos dos principais comandos e fluxos de trabalho do Git, úteis para consulta rápida durante o desenvolvimento.

\begin{itemize}
    \item Git Cheat Sheet oficial: \url{https://education.github.com/git-cheat-sheet-education.pdf}
    \item GitHub Git Cheat Sheet: \url{https://github.github.com/training-kit/downloads/github-git-cheat-sheet.pdf}
    \item Atlassian Git Cheat Sheet: \url{https://www.atlassian.com/git/tutorials/atlassian-git-cheatsheet}
\end{itemize}

Esses recursos complementam o conteúdo da apostila e ajudam a resolver dúvidas, explorar novas funcionalidades e aprimorar o uso do Git em projetos acadêmicos