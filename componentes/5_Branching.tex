\newpage
\section{Branching e Merging}

O uso de branches (ramificações) é um dos recursos mais poderosos do Git para organizar o desenvolvimento de projetos colaborativos. Branches permitem que diferentes funcionalidades, correções ou experimentos sejam desenvolvidos de forma independente, sem afetar o código principal do projeto. Após o desenvolvimento e testes, as alterações podem ser integradas ao projeto principal por meio do processo de merge (mesclagem).

\subsection{Introdução aos Branches}

Um branch é uma linha de desenvolvimento separada dentro do repositório Git. Por padrão, todo repositório possui um branch principal chamado \texttt{main} (ou \texttt{master}). Ao criar branches, você pode trabalhar em novas funcionalidades, corrigir bugs ou testar ideias sem interferir no código estável do projeto.

Branches são especialmente úteis em equipes, pois cada membro pode trabalhar em sua própria tarefa e, depois de pronta, integrar as alterações ao branch principal.

\subsection{Criando e Mudando de Branch}

Para criar um novo branch, utilize o comando:

\begin{lstlisting}[style=shellstyle]
git branch nome-da-branch
\end{lstlisting}

Para alternar para o branch criado:

\begin{lstlisting}[style=shellstyle]
git checkout nome-da-branch
\end{lstlisting}

Ou, de forma combinada:

\begin{lstlisting}[style=shellstyle]
git checkout -b nome-da-branch
\end{lstlisting}

Exemplo: Para desenvolver uma nova funcionalidade de simulação, crie um branch chamado \texttt{feature/simulacao}:

\begin{lstlisting}[style=shellstyle]
git checkout -b feature/simulacao
\end{lstlisting}

\subsection{Mesclando Branches}

Após finalizar o desenvolvimento em um branch, é necessário integrar as alterações ao branch principal. Isso é feito com o comando \texttt{merge}:

\begin{lstlisting}[style=shellstyle]
git checkout main
git merge feature/simulacao
\end{lstlisting}

O Git irá combinar as alterações do branch \texttt{feature/simulacao} ao branch \texttt{main}. Se não houver conflitos, o merge será realizado automaticamente.

\subsection{Resolvendo Conflitos de Merge}

Conflitos de merge ocorrem quando duas ou mais pessoas modificam a mesma linha de um arquivo em branches diferentes. O Git sinaliza o conflito e pede que o usuário escolha qual versão manter.

Ao realizar um merge e encontrar um conflito, o arquivo afetado exibirá marcações como:

\begin{verbatim}
<<<<<<< HEAD
Conteúdo da branch atual
=======
Conteúdo da branch a ser mesclado
>>>>>>> feature/simulacao
\end{verbatim}

Para resolver o conflito, edite o arquivo, escolha o conteúdo correto e remova as marcações. Depois, finalize o merge:

\begin{lstlisting}[style=shellstyle]
git add arquivo-afetado.ext
git commit -m "Resolve conflito de merge"
\end{lstlisting}

\subsection{Rebasing (Opcional)}

O rebase é uma alternativa ao merge para integrar alterações de um branch ao outro. Ele reorganiza o histórico de commits, tornando-o mais linear. O comando básico é:

\begin{lstlisting}[style=shellstyle]
git checkout feature/simulacao
git rebase main
\end{lstlisting}

O rebase é útil para manter um histórico limpo, mas deve ser usado com cautela em projetos colaborativos, pois pode reescrever o histórico de commits.

\subsection{Exemplo Prático: Fluxo de Trabalho com Branches de Funcionalidade}

Imagine que sua equipe está desenvolvendo um projeto de simulação de voo e precisa adicionar uma nova funcionalidade para calcular a altitude máxima.

\begin{enumerate}
    \item \textbf{Criar um branch para a funcionalidade:}
    \begin{lstlisting}[style=shellstyle]
git checkout -b feature/altitude-maxima
    \end{lstlisting}
    \item \textbf{Desenvolver e commitar as alterações:}
    \begin{lstlisting}[style=shellstyle]
git add simulacao.m
git commit -m "Adiciona cálculo de altitude máxima"
    \end{lstlisting}
    \item \textbf{Mesclar o branch ao principal:}
    \begin{lstlisting}[style=shellstyle]
git checkout main
git merge feature/altitude-maxima
    \end{lstlisting}
    \item \textbf{Resolver conflitos, se houver, e finalizar o merge:}
    \begin{lstlisting}[style=shellstyle]
git add simulacao.m
git commit -m "Resolve conflito e integra cálculo de altitude máxima"
    \end{lstlisting}
\end{enumerate}

Esse fluxo permite que diferentes funcionalidades sejam desenvolvidas em paralelo, testadas e integradas ao projeto principal de forma organizada e segura.


